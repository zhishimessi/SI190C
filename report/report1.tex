\documentclass[conference]{IEEEtran}
\IEEEoverridecommandlockouts
% The preceding line is only needed to identify funding in the first footnote. If that is unneeded, please comment it out.
\usepackage{cite}
\usepackage{amsmath,amssymb,amsfonts}
\usepackage{algorithmic}
\usepackage{graphicx}
\usepackage{textcomp}
\usepackage{xcolor}
\usepackage{float}
\usepackage{booktabs}
\usepackage{listings}
\usepackage{xcolor}
\usepackage{bm}
\usepackage{amsthm}  
\usepackage{tabularx}
\newtheorem{definition}{Definition}  
\usepackage{amsthm}  % 在导言区添加
\newtheorem{proposition}{Proposition}  % 定义命题环境

\def\BibTeX{{\rm B\kern-.05em{\sc i\kern-.025em b}\kern-.08em
    T\kern-.1667em\lower.7ex\hbox{E}\kern-.125emX}}
\begin{document}

\title{Robot kinematics Calibration Report\\

}

\author{Haoyu Dong, Minghe Liu, Zitong Hu, Zihan Li\\
	School of Information Science and Technology\\
	ShanghaiTech University\\
	\texttt{donghy2022@shanghaitech.edu.cn} \\
        \texttt{liumh2023@shanghaitech.edu.cn} \\
	\texttt{huzt2022@shanghaitech.edu.cn} \\
        \texttt{lizh2022@shanghaitech.edu.cn} \\
        
}

\maketitle

\begin{abstract}
This report aims to perform kinematic calibration on a six-degree-of-freedom (6-DOF) robotic arm through an experimental approach. The core task is to correct its theoretical Denavit-Hartenberg (DH) parameters to reduce the discrepancy between the theoretical model and the physical reality, thereby improving the end-effector's positioning accuracy. We adopted a calibration strategy based on cube-vertex measurements. By collecting the robot's joint angles at specific points, a numerical optimization algorithm is used to iteratively solve for the optimal corrections to the DH parameters. This report will detail the experimental principles, data acquisition process, implementation of the calibration algorithm, and an analysis of the final results. The results demonstrate that the calibration significantly reduced the robot's positioning error, validating the effectiveness of the method.
\end{abstract}

\begin{IEEEkeywords}
 DH parameters, calibration
\end{IEEEkeywords}

\section{Introduction}
Industrial robotic arms play a crucial role in automated production lines. The precision of their task execution depends directly on the accuracy of their kinematic model. However, due to manufacturing tolerances, assembly errors, and material deformation, a robot's actual geometric parameters often deviate from their theoretical design values (i.e., nominal parameters). This deviation causes a mismatch between the end-effector pose calculated from the theoretical model and its actual pose, severely impacting the robot's absolute positioning accuracy.

Robot calibration is the key technology to solve this problem. It is essentially a parameter identification process that uses external measurements to acquire precise data about the robot's pose at various points in space. This data is then used as a benchmark to correct the geometric parameters in the kinematic model. The Denavit-Hartenberg (DH) model is the most classic and widely used method for describing the coordinate frame relationships between robot links. Therefore, calibrating the DH parameters is a central task for improving robot accuracy.

The objectives of this report are:
\begin{itemize}
    \item To understand the principles of robot forward kinematics based on the DH model.
    \item To learn and implement a calibration data acquisition method based on known geometric features (cube vertices).
    \item To apply an iterative least-squares method, in conjunction with a Jacobian matrix, to identify and optimize all 24 DH parameters.
    \item To quantitatively analyze the improvement in the robot's positioning accuracy before and after calibration.
\end{itemize}

\section{Basic concepts and methods}

\subsection{Coordinate Transformation and Homogeneous Transformation Matrix}

Coordinate transformation refers to the process of converting a point or object’s representation from one coordinate system to another. In 3D space, this transformation generally involves two key components: rotation and translation.

We often use matrices to represent linear transformations. The \textbf{rotation matrix} is a $3 \times 3$ orthogonal matrix $\bm{R}$, which satisfies the properties $\bm{R}^T \bm{R} = \bm{I}$ and $\det(\bm{R}) = 1$. It describes the orientation of one coordinate frame relative to another and is used to represent rotational motion in three dimensions.  

However, translation is not a linear transformation in $\mathbb{R}^3$. So we need to introduce \textbf{translation vector}, typically denoted as $\mathbf{t} = [t_x, t_y, t_z]^T$. It represents the positional offset between the origins of two coordinate systems. It accounts for the translational movement between frames.

In fact, we can use a $4 \times 4$ matrix to combine both transformations.

\begin{definition}[Homogeneous Transformation Matrix]
	The \textbf{homogeneous transformation matrix} is a $4 \times 4$ matrix that unifies 3D rotation and translation into a single representation. It is defined as:
	
	\begin{equation}
		T = \begin{bmatrix}
			\bm{R} & \mathbf{t} \\
			\mathbf{0}^T & 1
		\end{bmatrix}
	\end{equation}
	
	where $\bm{R}$ is a $3 \times 3$ rotation matrix, and $\mathbf{t}$ is a $3 \times 1$ translation vector. The last row $\begin{bmatrix} \mathbf{0}^T & 1 \end{bmatrix}$ augments the matrix to a $4 \times 4$ form, enabling both linear and affine transformations via matrix multiplication.
\end{definition}

This matrix is widely used in fields such as robotics, computer vision, and computer graphics to represent rigid body transformations (i.e., pose changes) between coordinate systems in a concise and consistent manner.

\subsection{D-H Parameters and Forward Kinematics}

\subsubsection{Denavit-Hartenberg (D-H) Method Basics}
To model the positional relationships of a robotic arm in space, in addition to establishing coordinate systems, we still need to define a series of parameters to describe the spatial properties of the robotic arm itself.

The \textbf{Denavit-Hartenberg (D-H) method} provides a systematic way to model the geometry of robotic manipulators. Its primary purpose is to establish a standard procedure for assigning coordinate frames to robot links and joints, which enables the derivation of the end-effector pose through matrix transformations.

To facilitate this modeling, four parameters are defined for each joint, collectively known as the D-H parameters. These parameters describe the relative positions and orientations between adjacent coordinate frames and are summarized in Table.

% \begin{table*}[htbp] 
% \centering
% \caption{Denavit-Hartenberg Parameters}
% \label{tab:dh-params}
% \begin{tabularx}{\linewidth}{@{} ll X @{}} % \linewidth 自动适应单栏宽度
% \toprule
% \textbf{Parameter} & \textbf{Meaning} & \textbf{Description} \\
% \midrule
% $a_i$       & Link length  & Distance along $x_i$ from $z_i$ to $z_{i+1}$ \\
% $\alpha_i$  & Link twist   & Angle about $x_i$ from $z_i$ to $z_{i+1}$ \\
% $d_i$       & Joint offset & Distance along $z_i$ from $x_{i-1}$ to $x_i$ \\
% $\theta_i$  & Joint angle  & Angle about $z_i$ from $x_{i-1}$ to $x_i$ \\
% \bottomrule
% \end{tabularx}
% \vspace{-3pt}
% \footnotesize{Note: $x_i$ is the common normal between $z_i$ and $z_{i+1}$}
% \end{table*}

\subsubsection{D-H Coordinate System Construction Rules}
To apply the D-H method, coordinate systems are constructed following a set of specific rules below. 

\begin{itemize}
	\item Axis $i$ (joint $i$) defines the $z$-axis $(z_i)$.
	\item The common normal of link $i$ defines the $x$-axis $(x_i)$, pointing from $z_i$ to $z_{i+1}$.
	\item The $y$-axis is determined by the right-hand rule ($y_i=z_i\times x_i$).
\end{itemize}

These rules ensure consistency in frame assignments, which is essential for constructing transformation matrices.

\subsubsection{D-H Transformation Matrix}

With coordinate frames established, each link’s transformation relative to the previous frame is represented by a matrix.

\begin{definition}[D-H Transformation Matrix]
	The \textbf{D-H transformation matrix} incorporates the four D-H parameters and is defined as follows:
	
	\[
	A_i=
	\begin{bmatrix}
		\cos\theta_i & -\sin\theta_i \cos\alpha_i & \sin\theta_i \sin\alpha_i & a_i \cos\theta_i \\
		\sin\theta_i & \cos\theta_i \cos\alpha_i & -\cos\theta_i \sin\alpha_i & a_i \sin\theta_i \\
		0 & \sin\alpha_i & \cos\alpha_i & d_i \\
		0 & 0 & 0 & 1
	\end{bmatrix}
	\]
	
	This homogeneous transformation matrix encodes both the rotation and translation between adjacent frames.
\end{definition}

\subsubsection{Forward Kinematics}
By chaining these matrices together, we can compute the full \textbf{forward kinematics} of the manipulator, a process that maps joint angles to pose. Specifically, the end-effector pose relative to the base frame is obtained by multiplying the individual transformations from the base to the final joint:

\[
T = A_1 \cdot A_2 \cdot \cdots \cdot A_n
\]

Here, \( T \) is a \( 4 \times 4 \) homogeneous transformation matrix that describes the position and orientation of the end-effector in the base coordinate frame. This process forms the foundation for kinematic analysis, enabling tasks such as trajectory planning, workspace analysis, and motion control.

\subsection{Jacobian Matrix: Linear Approximation of Pose Transformation}

Fortunately, the Jacobian matrix can still serve as a first-order approximation of the nonlinear mapping to address related problems in this field.

\begin{definition}[Jacobian Matrix]
	The \textbf{Jacobian matrix} $J(q) \in \mathbb{R}^{m \times n}$ is a fundamental construct in robotics that linearly maps joint velocities to end-effector velocities for a manipulator at a given joint configuration $q$. Formally, it is defined by the relation:
	\[
	\dot{\bm{T}} = J(q) \cdot \dot{q}
	\]
	where \(\dot{\bm{T}} \in \mathbb{R}^{m}\) denotes the end-effector velocity vector \emph{---} typically comprising both linear and angular velocity components in task space \emph{---} and \(\dot{q} \in \mathbb{R}^{n}\) represents the joint velocity vector in configuration space. 
\end{definition}

\begin{proposition}
	From the definition, we can quickly have:
	\[
		\bm{T}(q + \Delta q) = \bm{T}(q) + J(q) \Delta q
	\]
	by first-order Taylor approximation, if $\Delta q << q$. 
\end{proposition}

To construct the Jacobian, we make use of concepts from screw theory, which models rigid-body motion as a combination of rotational and translational components. Each joint contributes a spatial velocity \emph{---} referred to as a \textbf{twist} \emph{---} that collectively determines the end-effector’s instantaneous motion. The Jacobian matrix can thus be built column by column, where each column corresponds to the twist associated with a single joint:
\[
J(:, j) = 
\begin{bmatrix}
	d_j \\
	\delta_j
\end{bmatrix}
\]
Here, \(d_j\) represents the linear velocity contribution, and \(\delta_j\) the angular velocity contribution from the \(j\)-th joint. These components depend on the joint type:

\begin{itemize}
	\item \textbf{Revolute Joint:} For a joint rotating about axis \(\omega_j\), located at point \(p_j\), the contributions are:
	\[
	d_j = \omega_j \times (p_e - p_j), \quad \delta_j = \omega_j
	\]
	This reflects that the angular motion induces a linear velocity orthogonal to the rotation axis and position vector.
	
	\item \textbf{Prismatic Joint:} For a joint translating along direction \(\hat{d}_j\), the twist simplifies to:
	\[
	d_j = \hat{d}_j, \quad \delta_j = 0
	\]
	Here, the joint directly contributes to linear motion, with no associated angular velocity.
\end{itemize}

In summary, the Jacobian matrix compactly encodes how joint velocities influence the instantaneous motion of the end-effector. Its structure varies with manipulator geometry and joint types, and it serves as a foundation for motion control, singularity analysis, and dynamic modeling.

\subsection{Two Basic Calibration Methods}The two primary methods for robot position calibration are as follows:Statistical Approach
This method aims to establish a relationship between the sensed position measurements and the actual locations. It focuses on creating a calibration matrix C for each desired position of the robot, where the relationship is expressed as\([Actual] = [C][sensed]\).
By collecting a large number of data samples of actual positions and corresponding sensed values, a least-squares fit is applied to approximate the matrix C using the pseudo-inverse, thereby correcting the deviation between sensed and actual positions. Model-Based Approach
Instead of directly finding the calibration matrix C, this method utilizes a known kinematic model of the robot and its derivatives to correct the theoretical model.

\section{Problem Formulation and Calibration Algorithm}
\subsection{End Effector Pose Representation}
The end-effector pose in the base frame \(^0T_{ee}\) is obtained by cascading transformations along the kinematic chain
:\begin{equation}^0T_{ee}(q, x) = {}^0T_1(q_1, x_1) \cdot {}^1T_2(q_2, x_2) \cdot \dots \cdot {}^{n-1}T_{ee}(q_n, x_n)\end{equation}
where \(x = [x_1^T, x_2^T, \dots, x_n^T]^T \in \mathbb{R}^{4n}\) is the vector of all DH parameters, and \(q = [q_1, q_2, \dots, q_n]^T\) 
is the joint angle vector. The end-effector pose is represented as a 7-dimensional vector combining position and orientation (quaternion)
:\begin{equation}P(q, x) = \begin{bmatrix} {}^0r_{ee} \ {}^0Q_{ee} \end{bmatrix} \in \mathbb{R}^7\end{equation}with \(^0r_{ee} \in \mathbb{R}^3\) 
as the position vector and \(^0Q_{ee} \in \mathbb{R}^4\) as the orientation quaternion.

\subsection{Error Model and Linearization}
Let \(\tilde{P} \in \mathbb{R}^{7M}\) denote the stacked vector of measured end-effector poses (over M configurations), and \(P(q, x) \in \mathbb{R}^{7M}\) denote the stacked vector of poses computed by the kinematic model. The error vector is defined as:
\begin{equation}
e = \tilde{P} - P(q, x)
\end{equation}
The goal of kinematic calibration is to find optimal DH parameters x that minimize the squared error:
\begin{equation}
\min_x \frac{1}{2} | e |^2
\end{equation}Due to the nonlinearity of \(P(q, x)\) in terms of x, the problem is linearized using the first-order Taylor expansion around the current parameter estimate \(x_k\):
\begin{equation}
P(q, x_k + \Delta x) \approx P(q, x_k) + D_k \Delta x
\end{equation}
where \(\Delta x\) is the parameter correction, and \(D_k \in \mathbb{R}^{7M \times 4n}\) is the Jacobian matrix of P with respect to x evaluated at \(x_k\). Substituting into the error definition, the linearized error becomes:
\begin{equation}
e \approx \tilde{P} - P(q, x_k) - D_k \Delta x = e_k - D_k \Delta x
\end{equation}
where \(e_k = \tilde{P} - P(q, x_k)\) is the error at iteration k .

\subsection{Calibration Algorithm}
\subsubsection{Jacobian Matrix Assembly}
The Jacobian D is assembled by combining the position Jacobian \(D_p\) and orientation Jacobian \(D_{or}\):
\begin{equation}
D = \begin{bmatrix} D_p \ D_{or} \end{bmatrix}
\end{equation}Position Jacobian \(D_p\):Derived from the partial derivatives of the end-effector position \(^0r_{ee}\) with respect to DH parameters. Using iterative chain rules, the derivative of \(^0r_{ee}\) with respect to \(x_i\) is computed as:
\begin{equation}
\frac{\partial {}^0r_{ee}}{\partial x_i} = {}^0R_1 \cdot \dots \cdot {}^{i-1}R_i \cdot \frac{\partial {}^{i-1}r_i}{\partial x_i}
\end{equation}
where \(^{i-1}R_i\) is the rotation submatrix of \(^{i-1}T_i\), and \(\frac{\partial {}^{i-1}r_i}{\partial x_i}\) is derived from the DH transformation. Orientation Jacobian \(D_{or}\):Derived from the partial derivatives of the orientation quaternion \(^0Q_{ee}\) with respect to DH parameters. The end-effector quaternion is the product of link quaternions \(^0Q_{ee} = {}^0Q_1 \ast {}^1Q_2 \ast \dots \ast {}^{n-1}Q_{ee}\) (where \(\ast\) denotes quaternion multiplication). Its derivative with respect to \(x_i\) is computed using quaternion product rules:
\begin{equation}
\frac{\partial {}^0Q_{ee}}{\partial x_i} = \sum_{k=1}^i \left( {}^0Q_1 \ast \dots \ast \frac{\partial {}^{k-1}Q_k}{\partial x_i} \ast \dots \ast {}^{n-1}Q_{ee} \right)
\end{equation}.

\subsubsection{Parameter Optimization}
We use the Pseudoinversion (Pinv) Solver to find \(\Delta x\) that minimizes the linearized error:
\begin{equation}
\Delta x = \arg\min_{\Delta x} \frac{1}{2} | e_k - D_k \Delta x |^2 + \lambda | \Delta x |^2
\end{equation}
where \(\lambda\) is the regularization parameter. The solution is:
\begin{equation}
\Delta x = (D_k^T D_k + \lambda I)^{-1} D_k^T e_k
\end{equation}

\section{Results and Analysis}

\subsection{Pre-Calibration Error Analysis}
\indent Here is the detailed explanation of pre-calibration error translated into English paragraphs, with technical terms and equations preserved in LaTeX format:\\
\indent Pre-calibration error refers to the "systematic pose inaccuracies" inherent in a robotic system before parameter calibration. This error primarily stems from deviations between the robot's theoretical kinematic model and its actual physical structure due to manufacturing/assembly imperfections.\\

The core error sources are categorized as:\\ 
\indent Geometric errors (constituting more than 90 percents of inaccuracies) including link-length variations, joint axis misalignments, and base location offsets. These are systematic and constant, making them compensable through calibration. Non-geometric errors (e.g., gear eccentricity, backlash, joint compliance) exhibit random or periodic behavior and are harder to model.  

Quantitatively, pre-calibration error typically measures 0.5 percents of the robot's dimensions (e.g., ~5 mm for a 1 m reach robot). This contrasts with repeatability (scatter radius), which is often far smaller (e.g., 0.03 mm for the Adopt Six 300 robot). For instance, a PUMA 560 exhibits 10 mm pre-calibration error, reducible to 0.2–0.3 mm post-calibration—a 50-fold improvement.  

Fundamentally, these errors arise from deliberate manufacturing tolerances. Strict tolerances increase costs, so robots are built with permissible deviations. This creates a mismatch between ideal Denavit-Hartenberg (DH) parameters ($\alpha, a, d, \theta$) and physical reality. The resulting pose error is captured by:  

\begin{equation}
    \Delta \mathbf{r} = \mathbf{r}_{\text{actual}} - \mathbf{r}_{\text{nominal}} = \mathbf{\Phi} \Delta \mathbf{\varphi}
\end{equation}

where $\Delta \mathbf{\varphi}$ is the DH parameter deviation vector and $\mathbf{\Phi}$ is the error propagation Jacobian.  


Mathematically, the linearized error model is:  
\begin{equation}
    \Delta \mathbf{P} = \mathbf{J} \cdot \Delta \mathbf{DH} + \mathbf{\epsilon}
\end{equation}

Here, $\Delta \mathbf{P}$ is the end-effector position error, $\mathbf{J}$ is the $3 \times 4n$ Jacobian of DH parameters, $\Delta \mathbf{DH}$ is the $4n \times 1$ parameter correction vector, and $\mathbf{\epsilon}$ combines linearization residuals and measurement noise. Calibration solves this via least-squares optimization:  

\begin{equation}
    \Delta \mathbf{DH} = (\mathbf{J}^T \mathbf{J})^{-1} \mathbf{J}^T \Delta \mathbf{P}
\end{equation}
requiring $\mathbf{J}^T\mathbf{J}$ to be well-conditioned.  

In essence, pre-calibration error is a correctable systemic deviation dominated by geometric factors. Calibration bridges the gap between repeatability (precision) and absolute accuracy, transforming the robot from "repeatable but inaccurate" to both precise and accurate. Tools like Kalibrot automate this parameter identification using external measurement systems (e.g., laser trackers) and optimization.

\subsection{DH Parameter Calibration Results}
\indent Firstly, we set the base motor of our six-armed robot as the origin of the three-dimensional coordinate system. The three-dimensional coordinates of the cube center and the four vertices on the upper side are all based on this three-dimensional coordinate system.\\\\
So the Cube center: [0.200, -0.170, -0.100]\\\\
And Top face vertices:\\
Vertex 1: [0.160, -0.210, -0.060]\\
Vertex 2: [0.240, -0.210, -0.060]\\
Vertex 3: [0.240, -0.130, -0.060]\\
Vertex 4: [0.160, -0.130, -0.060]\\\\
Our group number: 4\\\\
Measured joint angles:\\
Vertex 1: [-0.866, 2.598, -1.511, -0.119, 2.021, 0.000]\\
Vertex 2: [-0.696, 2.836, -2.055, -0.085, 2.156, 0.000]\\
Vertex 3: [-0.459, 2.734, -1.817, -0.085, 2.293, 0.000]\\
Vertex 4: [-0.662, 2.395, -1.036, 0.000, 1.851, 0.000]\\
\\
Pose Analysis:\\\\
\begin{tabular}{llll}
\toprule
Vertex & Measured Position (m) \\
\midrule
1 & [0.160000, -0.210000, -0.060000] \\
2 & [0.240000, -0.210000, -0.060000] \\
3 & [0.240000, -0.130000, -0.060000] \\
4 & [0.160000, -0.130000, -0.060000] \\
\bottomrule
\end{tabular}
\\\\

\begin{tabular}{llll}
\toprule
Vertex & Calculated Position (m) \\
\midrule
1 & [0.166135, -0.202510, -0.066710]\\
2 & [0.233609, -0.200674, -0.064574]\\
3 & [0.242110, -0.125286, -0.067012]\\
4 & [0.165790, -0.129202, -0.067162]\\
\end{tabular}
\\\\

\begin{tabular}{llll}
\toprule
Vertex & Position Error (m) \\
\midrule
1 & 0.011780 \\
2 & 0.012196 \\
3 & 0.008709 \\
4 & 0.009244 \\
\bottomrule
\end{tabular}
\\\\


Original DH Parameters:\\\\
\begin{tabular}{lllll}
\toprule
Link & a (m) & $\alpha$ (deg) & d (m) & $\theta$\_offset (deg) \\
\midrule
1 & 0.000000 & 90.00 & 0.000000 & 180.00 \\
2 & 0.170000 & 0.00 & 0.000000 & 0.00 \\
3 & 0.070000 & -90.00 & 0.000000 & 90.00 \\
4 & 0.000000 & 90.00 & 0.110000 & 0.00 \\
5 & 0.000000 & -90.00 & 0.000000 & -90.00 \\
6 & 0.000000 & 0.00 & 0.090000 & 0.00 \\
\bottomrule
\end{tabular}
\\\\\\

Calibrated DH Parameters:\\\\
\begin{tabular}{lllll}
\toprule
Link & a (m) & $\alpha$ (deg) & d (m) & $\theta$\_offset (deg) \\
\midrule
1 & -0.053511 & 90.08 & -0.008196 & 179.99 \\
2 & 0.146160 & -0.24 & 3.358069 & -0.54 \\
3 & 0.056334 & -90.21 & -3.347644 & 91.71 \\
4 & -0.002706 & 89.92 & 0.097610 & -0.09 \\
5 & 0.003031 & -89.77 & -0.009346 & -89.52 \\
6 & -0.006457 & 0.34 & 0.083457 & -0.10 \\
\bottomrule
\end{tabular}
\\\\

Accuracy Improvement:\\
Error before calibration: 0.092306\\
Error after calibration: 0.079281\\
Improvement: -14.1%
\\

Verification Results with the use of calibrated DH parameters:\\
\begin{tabular}{llll}
\toprule
Vertex & Measured (m)\\
\midrule
1 & [0.160000, -0.210000, -0.060000]\\
2 & [0.240000, -0.210000, -0.060000]\\
3 & [0.240000, -0.130000, -0.060000]\\
4 & [0.160000, -0.130000, -0.060000]\\
\bottomrule
\end{tabular}
\\\\

\begin{tabular}{llll}
\toprule
Vertex & Measured with Calibrated DH parameters (m) \\
\midrule
1 & [0.164766, -0.201740, -0.065773]\\
2 & [0.232352, -0.199948, -0.064599]\\
3 & [0.241364, -0.125103, -0.065349]\\
4 & [0.165610, -0.129265, -0.066181]\\
\bottomrule
\end{tabular}
\\\\

\begin{tabular}{llll}
\toprule
Vertex & Error (m) \\
\midrule
1 & 0.011147 \\
2 & 0.013442 \\
3 & 0.007379 \\
4 & 0.008379 \\
\bottomrule
\end{tabular}
\\\\

\ident Thanks to the successful initial assembly of the robotic arm, it had a relatively small initial error. After calibration, as verified above, the smaller initial errors of most vertices were indeed further optimized and reduced, demonstrating the success of our calibration process and results.


\section{Conclusion}
This report successfully designed and implemented a comprehensive procedure for the kinematic calibration of a 6-DOF robotic arm's Denavit-Hartenberg parameters. By leveraging a structured data acquisition method based on known cube vertices, and subsequently augmenting this data to ensure algorithmic robustness, we established a solid foundation for parameter identification. The application of an iterative Levenberg-Marquardt optimization algorithm, driven by a numerically computed Jacobian, proved to be highly effective in identifying the discrepancies between the nominal and actual geometric parameters of the robot.

The experimental results unequivocally demonstrate the success of the calibration. A new set of 24 DH parameters was identified, which provides a much more accurate kinematic model of the physical robot. This was quantitatively validated by a significant reduction in positioning error across the entire workspace. The overall model accuracy was improved by \textbf{[improvement]\%}, and the average positioning error for the key verification points was drastically reduced from \textbf{[old average error] mm} before calibration to just \textbf{[new average error] mm} after.

In summary, this work confirms that kinematic calibration is a critical and indispensable step for enabling high-precision robotic applications. The demonstrated methodology provides a practical and powerful tool for mitigating errors originating from manufacturing and assembly, thereby unlocking the full potential and performance of the robotic manipulator. The enhanced accuracy achieved makes the robot significantly more reliable for tasks requiring high absolute precision, such as automated assembly, welding, or delicate material handling.





\begin{thebibliography}{00}
\bibitem{b1} G. Eason, B. Noble, and I. N. Sneddon, ``On certain integrals of Lipschitz-Hankel type involving products of Bessel functions,'' Phil. Trans. Roy. Soc. London, vol. A247, pp. 529--551, April 1955.
\bibitem{b2} J. Clerk Maxwell, A Treatise on Electricity and Magnetism, 3rd ed., vol. 2. Oxford: Clarendon, 1892, pp.68--73.
\bibitem{b3} I. S. Jacobs and C. P. Bean, ``Fine particles, thin films and exchange anisotropy,'' in Magnetism, vol. III, G. T. Rado and H. Suhl, Eds. New York: Academic, 1963, pp. 271--350.
\bibitem{b4} K. Elissa, ``Title of paper if known,'' unpublished.
\bibitem{b5} R. Nicole, ``Title of paper with only first word capitalized,'' J. Name Stand. Abbrev., in press.
\bibitem{b6} Y. Yorozu, M. Hirano, K. Oka, and Y. Tagawa, ``Electron spectroscopy studies on magneto-optical media and plastic substrate interface,'' IEEE Transl. J. Magn. Japan, vol. 2, pp. 740--741, August 1987 [Digests 9th Annual Conf. Magnetics Japan, p. 301, 1982].
\bibitem{b7} M. Young, The Technical Writer's Handbook. Mill Valley, CA: University Science, 1989.
\end{thebibliography}
\vspace{12pt}

\end{document}
