\documentclass{article}
\usepackage{amsmath,amssymb,graphicx}
\usepackage{geometry}
\geometry{a4paper, margin=1in}

\title{Robotic Hand-Eye Calibration Report}
\author{Haoyu Dong, Minghe Liu, Zitong Hu, Zihan Li\\
	School of Information Science and Technology\\
	ShanghaiTech University\\
	\texttt{donghy2022@shanghaitech.edu.cn} \\
        \texttt{liumh2023@shanghaitech.edu.cn} \\
	\texttt{huzt2022@shanghaitech.edu.cn} \\
        \texttt{lizh2022@shanghaitech.edu.cn} \\
}
\date{2025.7.10}

\begin{document}

\maketitle

\section{Abstract}
This report presents the theory, implementation, and results of a robotic hand-eye calibration experiment. The calibration was performed using the Tsai-Lenz algorithm to determine the fixed transformation between a camera mounted on a robot's end-effector (eye-in-hand configuration) and the robot base. The mathematical foundations are derived in detail, followed by presentation of the experimental results and discussion.

\section{Introduction}
Hand-eye calibration is a fundamental problem in robotics that aims to determine the fixed transformation between a camera (eye) and a robot's end-effector (hand). This transformation is crucial for tasks that require visual feedback to guide robotic manipulation. The core equation governing this problem is:

\begin{equation}
A_i X = X B_i
\end{equation}

where:
\begin{itemize}
\item $A_i$ represents the transformation between robot base and end-effector for pose $i$
\item $B_i$ represents the transformation between camera and calibration object for pose $i$
\item $X$ is the unknown transformation between camera and end-effector (eye-in-hand) or camera and base (eye-to-hand)
\end{itemize}

\section{Mathematical Foundations}

\subsection{Homogeneous Transformations}
The transformations are represented as 4×4 homogeneous matrices:

\begin{equation}
H = \begin{bmatrix} R & t \\ 0 & 1 \end{bmatrix}
\end{equation}

where $R$ is a 3×3 rotation matrix and $t$ is a 3×1 translation vector.

\subsection{Rotation Representations}
Rotations can be represented in several forms:

\begin{enumerate}
\item \textbf{Axis-angle representation}: A rotation of angle $\theta$ about unit axis $k$
\item \textbf{Rotation vector}: $r = \theta k$
\item \textbf{Rodrigues' rotation formula}:

\begin{equation}
R = I + \sin\theta \cdot k_x + (1 - \cos\theta) \cdot k_x^2
\end{equation}

where $k_x$ is the skew-symmetric matrix of $k$.
\end{enumerate}

\subsection{Modified Rodrigues Parameters (MRP)}
\begin{equation}
P = 2 \sin\left(\frac{\theta}{2}\right) k
\end{equation}

\section{Tsai-Lenz Algorithm Derivation}
The Tsai-Lenz algorithm solves the hand-eye calibration problem in two steps: first for rotation, then for translation.

\subsection{Solving for Rotation}
From $A_i X = X B_i$, we extract the rotation part:

\begin{equation}
R_{A_i} R_x = R_x R_{B_i}
\end{equation}

\begin{enumerate}
\item Convert $R_{A_i}$ and $R_{B_i}$ to rotation vectors $r_{A_i}$ and $r_{B_i}$ using Rodrigues' formula:

\begin{equation}
\theta = \cos^{-1} \left( \frac{\text{trace}(R) - 1}{2} \right)
\end{equation}

\begin{equation}
k = \frac{1}{2 \sin \theta} \begin{bmatrix} R_{32} - R_{23} \\ R_{13} - R_{31} \\ R_{21} - R_{12} \end{bmatrix}
\end{equation}

\item Compute MRP for each rotation:

\begin{equation}
P_A = 2 \sin\left(\frac{\theta_A}{2}\right) k_A
\end{equation}
\begin{equation}
P_B = 2 \sin\left(\frac{\theta_B}{2}\right) k_B
\end{equation}

\item Solve for $P'_X$ using:

\begin{equation}
\text{skew}(P_A + P_B) \cdot P'_X = P_B - P_A
\end{equation}

where $\text{skew}(v)$ is the skew-symmetric matrix of vector $v$.

\item Compute final MRP:

\begin{equation}
P_x = \frac{2P'_x}{\sqrt{1 + |P'_x|}}
\end{equation}

\item Convert back to rotation matrix $R_x$:

\begin{equation}
R_x = \left( 1 - \frac{|P_x|^2}{2} \right) I + \frac{1}{2} \left( P_x P^T_x + \sqrt{4 - |P_x|^2} \, \text{Skew}(P_x) \right)
\end{equation}
\end{enumerate}

\subsection{Solving for Translation}
With $R_x$ known, solve the translation part:

\begin{equation}
(I - R_{A_i}) t_x = t_{A_i} - R_x t_{B_i}
\end{equation}

This forms a linear system that can be solved using least squares when multiple measurements are available.

\section{Experimental Results}
The hand-eye calibration was performed using 19 valid image poses (out of 20 attempted). The camera intrinsics were provided as:

\textbf{Camera matrix:}
\begin{equation}
K = \begin{bmatrix}
594.0418 & 0 & 310.17813 \\
0 & 593.92505 & 214.5701 \\
0 & 0 & 1
\end{bmatrix}
\end{equation}

\textbf{Distortion coefficients:}
\begin{equation}
[-0.4783, 0.3981, 0.000756, 0.0002098, -0.3071]
\end{equation}

The calibration results provided the transformation from camera to base:

\textbf{Rotation matrix:}
\begin{equation}
R = \begin{bmatrix}
-0.1768 & -0.6447 & -0.7437 \\
-0.8130 & 0.5215 & -0.2588 \\
0.5547 & 0.5589 & -0.6163
\end{bmatrix}
\end{equation}

\textbf{Translation vector:}
\begin{equation}
t = [0.0917, 0.2581, 1.6916]^T
\end{equation}

This transformation can be represented as the homogeneous matrix:

\begin{equation}
X = \begin{bmatrix}
-0.1768 & -0.6447 & -0.7437 & 0.0917 \\
-0.8130 & 0.5215 & -0.2588 & 0.2581 \\
0.5547 & 0.5589 & -0.6163 & 1.6916 \\
0 & 0 & 0 & 1
\end{bmatrix}
\end{equation}

\section{Discussion and Conclusion}
The experiment successfully implemented the Tsai-Lenz algorithm for hand-eye calibration. The results show a reasonable transformation between the camera and robot base, with the z-component of translation (1.69m) suggesting the camera was mounted at some distance from the base.

Key observations:
\begin{itemize}
\item The algorithm successfully processed 19 out of 20 images, demonstrating robustness to occasional detection failures.
\item The rotation matrix satisfies orthogonality conditions ($R^T R \approx I$) with minor numerical errors.
\item The translation vector appears physically plausible for a robotic setup.
\end{itemize}

Potential improvements:
\begin{itemize}
\item Using more image poses could improve accuracy
\end{itemize}

This calibration provides the essential transformation needed for vision-guided robotic operations, enabling accurate mapping between camera coordinates and robot coordinates.

\section*{References}
\begin{enumerate}
\item R. Y. Tsai and R. K. Lenz, "A new technique for fully autonomous and efficient 3D robotics hand/eye calibration," IEEE Transactions on Robotics and Automation, 1989.
\item F. C. Park and B. J. Martin, "Robot sensor calibration: solving AX=XB on the Euclidean group," IEEE Transactions on Robotics and Automation, 1994.
\item Lecture notes: S1190C Robotics Practice Course, 2025 Summer @ SIST
\end{enumerate}

\end{document}
